\documentclass[12pt]{article}
\title{Case Study}
\usepackage{amsmath} %dfrac

\begin{document}


\section{Methodology}

Gaisler's GRSIM LEON3 simulator was used to estimate the number of
UCBs of a task. The estimation took four steps. An initial run to
determine the total number of instructions for a complete run of the
task. Then dividing the task into uniform intervals. The cache state
at the end of each interval was recorded. Lastly, the captured cache
state was compared to determine the UCBs between pairs of program points.

Tasks were selected from the MRTC WCET Benchmark suite. They were
modifiend and built using Gaisler's LEON3 compiler. Then run on the
LEON3 GRSIM simulator. Using the simulator, the total number of
instructions required for a complete execution of a task was recorded
as ${IC}$.

The tasks were divided into ${v}$ intervals of ${\dfrac{IC}{v}}$
instructions. Yielding ${v}$ program points ${\{P_1, P_2, \ldots,
  P_v\}}$. Each program point corresponds to the last instruction of
each interval. Using the simulator, the complete instuction and data
caches were captured at each program point. As nomenclature, the complete
instruction and data cache captured at program point ${P_j}$ will be
referred to as ${C_j}$.


Estimating the UCBs between two program points ${P_j}$ and ${P_k}$ was
done by taking the intersection of cache states between the two
points: 

\begin{equation*}
  UCB(P_j, P_k) = \left\vert\bigcap_{i=j}^{k}C_i\right\vert
\end{equation*}

The reasoning
is simple, those cache entries common to all program points ${P_j}$
through ${P_k}$ are the only UCB candidates. Counting those entries
serves as an upper bound on the UCBs between ${P_j}$ and ${P_k}$.

\section{Results}

The results are presented as a comparison between the method described
herein and the Bertonga approach. For a program point ${P_j}$ the
Bertogna approach defines the UCBs (and therefor the CRPD) as:

\begin{equation*}
  max\{ UCB(P_i, P_j) \vert i < j \}
\end{equation*}

To determine the maximum benefit of the new approach, the best case
scenario is considered. When the preemption point is selected with the
fewest number of UCBs based upon previous preemption.  For ${P_j}$ the
determination is made by:

\begin{equation*}
  min\{ UCB(P_i, P_j) \vert i < j \}
\end{equation*}

The difference between these values is calculated for all program
points and then averaged. For the task set selected from the MRTC
Benchmark Suite, the average reduction in UCBs for instruction caches
is 41.40\% and data caches 2.28\%.

\section{References}

MRTC Benchmarks 
http://www.mrtc.mdh.se/projects/wcet/benchmarks.html


Gaisler GRSIM 
http://gaisler.com/index.php/products/simulators/grsim


Gaisler Compiler 
http://gaisler.com/index.php/downloads/compilers


\end{document}


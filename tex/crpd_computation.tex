\vspace{-5pt}
\section{CRPD Computation}\label{sec:crpd_computation}

While existing research has focused on computing the upper bounds on cache related preemption delay (CRPD), our approach achieves higher accuracy by computing the \emph{loaded cache blocks} (LCBs), as defined below, due to higher priority task preemption by using the set of potential/chosen preemption locations.  The sets of various cache blocks are represented as sets of integers.  We employ the CRPD terms UCB and ECB as defined by Lee et al.~\cite{lee:98} and by Altmeyer and Burguiere~\cite{altmeyer:11c}.
%Since the variables P and Q in the original are overloaded here, we modify the definition alternatively using %basic block identifiers.

\begin{definition}
\textbf{Useful Cache Block (UCB)}: A memory block $m$ is called a useful cache block at program point \begin{math}\delta_{i}^{j_{1}}\end{math}, if \\(a) $m$ may be cached at \begin{math}\delta_{i}^{j_{1}}\end{math} and \\(b) $m$ may be reused at program point \begin{math}\delta_{i}^{j_{2}}\end{math} that may be reached from \begin{math}\delta_{i}^{j_{1}}\end{math} without eviction of $m$ on this path.
\end{definition}

%\Leftrightarrow
\noindent More formally, cache block \begin{math}m \in \textit{UCB}(\tau_{i})\textrm{ if and only if } \tau_{i}\end{math} has $m$ as a useful cache block in some cache-set $s$.  Note this definition of \textit{UCB} embodies a task level view.  Cache block \begin{math}m \in \textit{UCB}_{\textrm{out}}(\delta_{i}^{j})\textrm{ if and only if }\delta_{i}^{j}\end{math} has $m$ as a useful cache block in some cache-set $s$ where
\begin{equation}\label{eqn:ucb-task}
    \textit{UCB}(\tau_{i}) = \bigcup_{\delta_{i}^{j} \in \tau_{i}} \textit{UCB}_{\textrm{out}}(\delta_{i}^{j}).
\end{equation}
\noindent The notation \begin{math}\textit{UCB}_{\textrm{out}}(\delta_{i}^{j})\end{math} is the set of useful cache blocks cached in task \begin{math}\tau_{i}\end{math} post-execution of basic block \begin{math}\delta_{i}^{j}\end{math}.  Similarly, the notation \begin{math}\textit{UCB}_{in}(\delta_{i}^{j})\end{math} is the set of useful cache blocks cached in task \begin{math}\tau_{i}\end{math} pre-execution of basic block \begin{math}\delta_{i}^{j}\end{math}.

\begin{definition}
\textbf{Evicting cache block (ECB)}: A memory block $m$ of the preempting task is called an evicting cache block, if it may be accessed during the execution of the preempting task.
\end{definition}

%\Leftrightarrow
\noindent Cache block \begin{math}m \in \textit{ECB}(\tau_{k})\textrm{ if and only if }\tau_{k}\end{math} may evict $m$ in some cache-set $s$.  Note this definition of \textit{ECB} also embodies a task level view.  Cache block \begin{math}m \in \textit{ECB}(\delta_{i}^{j})\textrm{ if and only if } \delta_{k}^{j}\end{math} may evict $m$ in some cache-set $s$ where
\begin{equation}\label{eqn:ecb-task}
    \textit{ECB}(\tau_{k}) = \bigcup_{\delta_{k}^{j} \in \tau_{k}} \textit{ECB}(\delta_{k}^{j}).
\end{equation}
\noindent The notation \begin{math}\textit{ECB}(\delta_{k}^{j})\end{math} is the set of evicting cache blocks accessed in task \begin{math}\tau_{k}\end{math} during execution of basic block \begin{math}\delta_{k}^{j}\end{math}.  In order to determine which cache blocks may be reloaded once preemption occurs, we introduce the notion of an accessed useful cache block (AUCB).
%
%Altmeyer and Burguiere~\cite{altmeyer:11c} introduced a refinement of \textit{UCB} to capture extrinsic cache %blocks evicted due to inter-task preemption thereby removing the cache blocks evicted via intra-task %execution.
%
%\begin{definition}
%\textbf{Definitely Cached Useful Cache Block (DC-UCB)}: A memory block $m$ is called a definitely cached useful %cache block at program point \begin{math}\delta_{i}^{j_{1}}\end{math}, if \\(a) $m$ must be cached at %\begin{math}\delta_{i}^{j_{1}}\end{math} and \\(b) $m$ may be reused at program point %\begin{math}\delta_{i}^{j_{2}}\end{math} that may be reached from \begin{math}\delta_{i}^{j_{1}}\end{math} %without eviction of $m$ on this path.
%\end{definition}
%
%\noindent More formally, cache-set \begin{math}s \in \textit{DC-UCB}(\tau_{i}) if and only if %\tau_{i}\end{math} has a definitely cached useful cache block in cache-set $s$.  Note this definition of %\textit{DC-UCB} also embodies a task level view.  Cache-set \begin{math}s \in %\textit{DC-UCB}_{\textrm{out}}(\delta_{i}^{j})\textit{ if and only if }\delta_{i}^{j}\end{math} has a definitely cached useful %cache block in cache-set $s$ where
%
%\begin{equation}\label{eqn:dcucb-task}
%    \textit{DC-UCB}(\tau_{i}) = \bigcup_{j} \textit{DC-UCB}_{\textrm{out}}(\delta_{i}^{j}).
%\end{equation}
%
%\noindent The notation
%\begin{math}\textit{DC-UCB}_{\textrm{out}}(\delta_{i}^{j})\end{math} is the set of definitely cached useful cache blocks %cached in task \begin{math}\tau_{i}\end{math} post execution of basic block %\begin{math}\delta_{i}^{j}\end{math}.  Similarly, the notation %\begin{math}\textit{DC-UCB}_{in}(\delta_{i}^{j})\end{math} is the set of definitely cached useful cache blocks %cached in task \begin{math}\tau_{i}\end{math} pre-execution of basic block %\begin{math}\delta_{i}^{j}\end{math}.

%To capture the notion of what we are attempting to count, consider the set of replaced cache blocks (RCBs).
%
%\begin{definition}
%\textbf{Replaced cache block (RCB)}: A memory block of the preempted task is called a replaced cache block, if %it is replaced during the execution of a preempting task.
%\end{definition}
%
%\noindent \begin{math}\textit{RCB}_{\textrm{out}}(\delta_{i}^{j},\tau_{k})\end{math} denotes preemption of task %\begin{math}\tau_{i}\end{math} by task \begin{math}\tau_{k}\end{math} occurring immediately after basic block %\begin{math}\delta_{i}^{j}\end{math}, where task \begin{math}\tau_{i}\end{math} at basic block location %\begin{math}\delta_{i}^{j}\end{math} is a potential/selected preemption point in our preemption point placement %algorithm described later.  We compute the set of replaced cache blocks (RCBs) as follows:
%\begin{equation}\label{eqn:rcb-formula}
%    \textit{RCB}_{\textrm{out}}(\delta_{i}^{j},\tau_{k}) = \textit{DC-UCB}_{\textrm{out}}(\delta_{i}^{j}) \cap
%    \textit{ECB}(\tau_{k}).
%    \textit{RCB}_{\textrm{out}}(\delta_{i}^{j},\tau_{k}) = \textit{UCB}_{\textrm{out}}(\delta_{i}^{j}) \cap %\textit{ECB}(\tau_{k}).
%\end{equation}
%
%\noindent where \begin{math}\textit{DC-UCB}_{\textrm{out}}(\delta_{i}^{j})\end{math} is the set of definitely cached %useful cache blocks for task \begin{math}\tau_{i}\end{math} post basic block
%\noindent where \begin{math}\textit{UCB}_{\textrm{out}}(\delta_{i}^{j})\end{math} is the set of useful cache blocks for %task \begin{math}\tau_{i}\end{math} post basic block \begin{math}\delta_{i}^{j}\end{math} execution; and %\begin{math}\textit{ECB}(\tau_{k})\end{math} is the set of extrinsic evicting cache blocks of the preempting %task \begin{math}\tau_{k}\end{math}.
%\noindent
%If a cache block is evicted there are two options in terms of where the preemption overhead cost can be paid: %1) the cost can be paid where the next memory accesses occur; or 2) The cost can also be paid where the %preemptions occur.  In order to determine which cache blocks are reloaded once preemption occurs, we introduce %the notion of an accessed useful cache block (AUCB).
%If a cache block is evicted the cost is paid where the preemptions occur.

\begin{definition}
\textbf{Accessed useful cache block (AUCB)}: A memory block of the preempted task is called an accessed useful cache block if it may be accessed during the execution of a basic block \begin{math}\delta_{i}^{j}\end{math} for task \begin{math}\tau_{i}\end{math}.
\end{definition}

%\noindent The term \begin{math}\textit{AUCB}_{\textrm{out}}(\delta_{i}^{j})\end{math} represents the definitely cached %useful cache blocks (DC-UCBs) accessed by task \begin{math}\tau_{i}\end{math} at during execution of basic %block
\noindent The term \begin{math}\textit{AUCB}_{\textrm{out}}(\delta_{i}^{j})\end{math} represents the useful cache blocks (UCBs) accessed by task \begin{math}\tau_{i}\end{math} during execution of basic block at location \begin{math}\delta_{i}^{j}\end{math}. The definition of \textit{AUCB} is introduced to capture the set of task accessed memory at a specific basic block location subsequently used in the calculation of blocks that must be reloaded when task preemptions occur.  We compute the set of accessed useful cache blocks (AUCBs) as follows:
\begin{equation}\label{eqn:aucb-formula}
%    \textit{AUCB}_{\textrm{out}}(\delta_{i}^{j}) = \textit{DC-UCB}_{\textrm{out}}(\delta_{i}^{j}) \cap
%    \textit{ECB}(\delta_{i}^{j}).
    \textit{AUCB}_{\textrm{out}}(\delta_{i}^{j}) = \textit{UCB}_{\textrm{out}}(\delta_{i}^{j}) \cap \textit{ECB}(\delta_{i}^{j}).
\end{equation}
%\noindent where \begin{math}\textit{DC-UCB}_{\textrm{out}}(\delta_{i}^{j})\end{math} is the set of definitely cached %useful cache blocks for task \begin{math}\tau_{i}\end{math} post basic block
\noindent where \begin{math}\textit{UCB}_{\textrm{out}}(\delta_{i}^{j})\end{math} is the set of useful cache blocks for task \begin{math}\tau_{i}\end{math} post basic block \begin{math}\delta_{i}^{j}\end{math} execution; and \begin{math}\textit{ECB}(\delta_{i}^{j})\end{math} denotes the set of cache blocks accessed in task \begin{math}\tau_{i}\end{math} during execution of basic block \begin{math}\delta_{i}^{j}\end{math}. It is important to note that only cache block evictions due to preemption are considered, as intrinsic cache misses are captured as part of WCET analysis in the term \begin{math}C_{i}^{NP}\end{math}.  Using the previously defined terms, we may now define and explicitly compute the cache blocks that are re-loaded due to preemption which are called loaded cache blocks (LCBs).

\begin{definition}
\textbf{Loaded cache block (LCB)}: A memory block of the preempted task is called a loaded cache block, if it may be re-loaded during the non-preemptive region (i.e., within a series of basic blocks with no preemptions) immediately following a preemption.
\end{definition}
\noindent
\begin{math}\textit{LCB}(\delta_{i}^{curr},\delta_{i}^{next})\end{math} denotes the set of cache blocks re-loaded during execution of the non-preemptive region between basic block \begin{math}\delta_{i}^{curr}\end{math} and basic block \begin{math}\delta_{i}^{next}\end{math}, resulting from preemption of task \begin{math}\tau_{i}\end{math}, where basic block location \begin{math}\delta_{i}^{curr}\end{math} and \begin{math}\delta_{i}^{next}\end{math} are the potential/selected preemption point and next potential/selected preemption point respectively.  In our model, a preemption point located at basic block \begin{math}\delta_{i}^{curr}\end{math} occurs at the edge between \begin{math}\delta_{i}^{curr}\end{math} and \begin{math}\delta_{i}^{curr+1}\end{math}.

The definition of \textit{LCB} is introduced to capture the set of reloaded cache memory at a specific basic block as a function of the current and next selected preemption points.  Here, we account for the overhead within a non-preemptive region for reloading UCBs that could have potentially been evicted by the preemption occurring immediately after $\delta_i^{curr}$ and used by some basic block prior to the preemption occurring immediately after $\delta_i^{next}$.
%As previously stated, there are two options in terms of where the cache block reload cost can be paid: 1) the cost can be paid where the next memory %accesses occur; or 2) The cost can also be paid where the preemptions occur.  In this paper, we use the cost paid where the preemptions occur as per %below.
%This results in two similar formulations for computing loaded cache blocks (LCBs) as per below.
%\newline
%\newline
%\noindent\textbf{Preemption Cost Paid Where Preemptions Occur}
\begin{equation}\label{eqn:lcb-formula-1}
\begin{split}
%    \lefteqn{\textit{LCB}(\delta_{i}^{curr},\delta_{i}^{next},\tau_{k}) =} & \\
%    &\left[\textit{DC-UCB}_{\textrm{out}}(\delta_{i}^{curr})\ \cap \left[\cup_\lambda
%    \textit{AUCB}_{\textrm{out}}(\delta_{i}^{\lambda})\right]\right] \cap \textit{ECB}(\tau_{k})\\
    \lefteqn{\textit{LCB}(\delta_{i}^{curr},\delta_{i}^{next}) =} & \\
    &\left[\textit{UCB}_{\textrm{out}}(\delta_{i}^{curr})\ \cap \left[\cup_{\nu \in \lambda} \textit{AUCB}_{\textrm{out}}(\delta_{i}^{\nu})\right]\right] \cap [\cup_{\tau_{k}\in hp_(i)} \textit{ECB}(\tau_{k})]\\
\end{split}
\end{equation}
%
%\newline
%\noindent
%where
%\vspace{-10pt)
\begin{equation*}\label{eqn:lcb-formula-1b}
    \textrm{where }\lambda \stackrel{\text{def}}{=} \{ \nu|\nu \in [ \textit{curr+1}, \textit{curr+2}, \ldots, \textit{next} ] \}
\end{equation*}
%
\noindent
%\begin{math}\textit{AUCB}_{\textrm{out}}(\delta_{i}^{x})\end{math} represents the accessed useful cache blocks or memory of interest accessed by the preempted %task \begin{math}\tau_{i}\end{math} at basic block \begin{math}\delta_{i}^{x}\end{math};
\begin{math}\delta_{i}^{curr}\end{math} represents the current selected preemption point where \begin{math}\delta_{i}^{curr} \in \rho_{i}\end{math} and \begin{math}\delta_{i}^{next}\end{math} represents the next selected preemption point where \begin{math}\delta_{i}^{next} \in \rho_{i}\end{math}, and  $\rho_i \subseteq \{\delta_i^0, \delta_i^1,  \ldots, \delta_i^{N_i}\}$ is an ordered set by ascending index of selected preemption points for task \begin{math}\tau_{i}\end{math}:
\begin{equation*}\label{eqn:pp-set_b}
\begin{split}
   \rho_{i} \stackrel{\text{def}}{=} \{\delta_{i}^{m}|&\delta_{i}^{m} \text{ is a selected preemption point of task } \tau_{i}\ \\ &\wedge m \in [0, 1, 2, \ldots, N_{i}]\}
\end{split}
\end{equation*}
\noindent
This formula for \begin{math}\textit{LCB}(\delta_{i}^{curr},\delta_{i}^{next})\end{math} results in the accounting of loaded cache blocks where the preemption occurs.  Note that this notation assumes a sequential basic block structure. Once we have the set of cache blocks that must be re-loaded due to preemption, the CRPD related preemption overhead may be computed as shown below.
%\newline
%\newline
%\noindent\textbf{Preemption Cost Paid Where Memory Reloads Occur}
%\begin{equation}\label{eqn:lcb-formula-2}
%\begin{split}
%    \lefteqn{\textit{LCB}(\delta_{i}^{j},\tau_{k}) =} & \\
%    &\left[\cup_\varphi \left[\textit{RCB}_{\textrm{out}}(\delta_{i}^{\varphi},\tau_{k}) \setminus \cup_\lambda
%    \textit{AUCB}_{\textrm{out}}(\delta_{i}^{\lambda})\right] \cap \textit{AUCB}_{\textrm{out}}(\delta_{i}^{j})\right]\\
%\end{split}
%\end{equation}
%\noindent
%where
%\begin{equation*}\label{eqn:lcb-formula-2b}
%\varphi \stackrel{\text{def}}{=} \{\kappa | (\kappa \geq \delta_{i}^{1} \wedge \kappa < \delta_{i}^{j})\}
%\end{equation*}
%\begin{equation*}\label{eqn:lcb-formula-2c}
%    \lambda \stackrel{\text{def}}{=} \{\nu | (\nu > \delta_{i}^{\varphi} \wedge \nu \leq
%     min(\delta_{i}^{j},\rho_{next}(\delta_{i}^{\varphi})))\}
%\end{equation*}
%
%\noindent
%Using the term \begin{math}\textit{LCB}(\delta_{i}^{j},\tau_{k})\end{math}, we can compute the preemption cost %as a function of the current and next selected preemption point:
%\begin{equation}\label{eqn:lcb-formula-3}
%\textit{LCB}(\delta_{i}^{curr},\delta_{i}^{next},\tau_{k}) = [\cup_\lambda %\textit{LCB}(\delta_{i}^{\lambda},\tau_{k})]
%\end{equation}
%
%\newline
%\noindent
%where
%\begin{equation*}\label{eqn:lcb-formula-1b}
%    \lambda \stackrel{\text{def}}{=} \{ \nu|\nu \in [ \delta_{i}^{curr} \text{,} \delta_{i}^{next} ] \}
%\end{equation*}
%
%\noindent
%\begin{math}\textit{AUCB}_{\textrm{out}}(\delta_{i}^{x})\end{math} represents the accessed useful cache blocks or memory %of interest accessed by the preempted task \begin{math}\tau_{i}\end{math} at basic block %\begin{math}\delta_{i}^{x}\end{math}; and \begin{math}\delta_{i}^{j}\end{math} represents the current basic %block location.  This formula for \begin{math}\textit{LCB}(\delta_{i}^{j},\tau_{k})\end{math} results in the %accounting of loaded cache blocks where the reload occurs.  The algorithms for optimal preemption placement for %linear basic block structure use the preemption cost paid where The preemptions occur formula.
%Once we have the set of cache blocks that must be re-loaded due to preemption, the CRPD related preemption %overhead may be computed as per below.
\begin{equation}\label{eqn:crpd-formula}
    \gamma_{i}(\delta_{i}^{curr},\delta_{i}^{next}) = | \textit{LCB}(\delta_{i}^{curr},\delta_{i}^{next}) | \cdot BRT.
\end{equation}
\noindent
where BRT is the cache block reload time; and \begin{math}\textit{LCB}(\delta_{i}^{curr},\delta_{i}^{next})\end{math} represents the loaded cache blocks or memory accessed by the preempted task \begin{math}\tau_{i}\end{math} at basic block \begin{math}\delta_{i}^{curr}\end{math} caused by higher priority preempting tasks.
%\newline
%\noindent
To illustrate our approach for computing LCBs, a suitable example is presented in the appendix due to space limitations here.
%\vspace{-10pt}

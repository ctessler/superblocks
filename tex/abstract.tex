\begin{abstract}\label{sec:abstract}

Schedulability analysis for real-time systems has been the subject of prominent research over the past several decades.  One of the key foundations of schedulability analysis is an accurate worst case execution time (WCET) measurement for each task.  In real-time systems supporting preemption, the cache related preemption delay (CRPD) can represent a significant component (up to 44\% as documented in research literature) ~\cite{pellizzoni:07} ~\cite{pellizzoni:08} ~\cite{pellizzoni:11} of variability to overall task WCET.  Several methods have been employed to calculate CRPD with significant levels of pessimism that may result in a task set erroneously declared as non-schedulable. Furthermore, they do not take into account that CRPD cost is inherently a function of where preemptions actually occur.  Our approach for computing CRPD is more accurate in the sense that cache state reflects which cache blocks and the specific program locations where they are reloaded.

Limited preemption models attempt to minimize preemption overhead (CRPD) by reducing the number of allowed preemptions and/or allowing preemption at program locations where the CRPD effect is minimized.  These algorithms rely heavily on accurate CRPD measurements or estimation models in order to identify an optimal set of preemption points.  Our approach improves the effectiveness of limited optimal preemption point placement algorithms by leveraging the enhanced CRPD computation to more accurately model task WCET and maximize schedulability as compared to existing preemption point placement approaches.  We propose a revised optimal preemption point placement algorithm suing dynamic programming.  Lastly, we will demonstrate using a case study and synthetically generated task sets, improved task set schedulability and optimal preemption point placement using the more accurate CRPD and associated task WCET values.

\end{abstract} 
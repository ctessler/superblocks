\section{Integrated WCET/CRPD Calculation}\label{sec:schedulability_analysis}

The modified preemption cost as a function of the current and next preemption points and the preempting task \begin{math}\tau_{k}\end{math} is given by:
\begin{equation}\label{eqn:prempt-cost}
\begin{split}
    \xi(\delta_{curr}^{i},\delta_{next}^{i},\tau_{k})\ =\ &\gamma(\delta_{curr}^{i},\delta_{next}^{i},\tau_{k}) + \pi + \sigma + \\ &\eta(\gamma(\delta_{curr}^{i},\delta_{next}^{i},\tau_{k})).
\end{split}
\end{equation}

 \noindent
 where \begin{math}\pi\end{math} is the pipeline cost, \begin{math}\sigma\end{math} is the scheduler processing cost, and \begin{math}\eta()\end{math} is the front side bus contention resulting from the cache reload interference as described in~\cite{pellizzoni:07} ~\cite{pellizzoni:08} ~\cite{pellizzoni:11}.  Commensurate with our worst case analysis, the preemption overhead cost as a function of the current and next selected preemption points is:
\begin{equation}\label{eqn:prempt-cost}
    \xi(\delta_{curr}^{i},\delta_{next}^{i}) = \textit{max}_{k \in hp(i)} [ \xi(\delta_{curr}^{i},\delta_{next}^{i},\tau_{k})].
\end{equation}

The output of our algorithm is an optimal set of preemption points subject to the maximum allowable non-preemption region \begin{math}Q_{i}\end{math}.  The optimal set of preemption points obtained using the enhanced accuracy of our preemption cost computation is used to calculate each task's WCET given by:
\begin{equation}\label{eqn:wcet-cost}
   C_{i} = B_{0,N_{i}}^{i}(\rho^{i}) = C_{i}^{NP} + \sum_{m=1}^{|\rho^{i}|-1} [\xi(\rho_{m}^{i},\rho_{m+1}^{i})].
\end{equation}

\noindent
where \begin{math}\rho^{i}\end{math} is the set of selected preemption points for task \begin{math}\tau_{i}\end{math}:
\begin{equation}\label{eqn:pp-set}
\begin{split}
   \rho^{i} \stackrel{\text{def}}{=} \{\delta_{m}^{i}|&\delta_{m}^{i} \text{ is a selected preemption point of task } \tau_{i}\ \wedge \\ &m \in [1,N_{i}]\}
\end{split}
\end{equation}

\noindent
and \begin{math}B_{k,m}^{i}\end{math} is the WCET with preemption overhead of basic blocks $k$ to $m$ of task \begin{math}\tau_{i}\end{math} as given by:
\begin{equation}\label{eqn:bbkwcet-cost}
\begin{split}
   B_{k,m}^{i}(\rho^{i}) = \min_{\delta_{j}^{i}} \Big[&B_{j-1}^{i}(\rho^{i}(\delta_{j-1}^{i})) + \xi(\delta_{j}^{i},\rho_{next}^{i}(\delta_{j}^{i})) + \\ &\sum_{n=k}^{m}b_{n}^{i}\Big].
\end{split}
\end{equation}

\noindent
where the term \begin{math}\rho_{next}^{i}(\delta_{j}^{i})\end{math} is the next potential/selected preemption point from basic block \begin{math}\delta_{j}^{i}\end{math}:
\begin{equation}\label{eqn:ppnext-set}
\begin{split}
   \rho_{next}^{i}(\delta_{k}^{i}) \stackrel{\text{def}}{=} \{\delta_{j}^{i}|&\delta_{j}^{i} = \rho_{m+1}^{i} \wedge \delta_{j}^{i} = \rho_{m+1}^{i} \\ &\textrm{ for some } m \in [1,N_{i}-1]\}
\end{split}
\end{equation}

\noindent
The maximum blocking time \begin{math}\beta_{i}\end{math} that each task may tolerate utilizes the computed task WCET parameter \begin{math}C_{i}\end{math}.  The method for obtaining the maximum blocking time is eloquently summarized for the Earliest Deadline First (EDF) and Fixed Priority (FP) scheduling by Bertogna et. al. algorithms~\cite{bertogna:11}~\cite{bertogna:10}.  The circular dependency between the maximum blocking time \begin{math}\beta_{i}\end{math} and task WCET \begin{math}C_{i}\end{math} parameters suggests an iterative approach to allow the two parameters to convergence to a steady state.  One such iterative approach contains the following steps as given in Algorithm~\ref{alg:iterative-schedulability-optimal-ppp}.

%\begin{spacing}{2.0}
{\fontsize{10}{10}\selectfont
\begin{algorithm}
\caption{Iterative Schedulability and Preemption Point Placement Algorithm}
\label{alg:iterative-schedulability-optimal-ppp}
\begin{algorithmic}[1]
\small
\State{Start with a task system that may or may not be feasible.}
\State{Assume the CRPD of the task system is initially zero.}
%\Repeat
    \Repeat
        \State\begin{varwidth}[t]{\linewidth}
        Run the Baruah algorithm to obtain the maximum \par
        \hskip\algorithmicindent non-preemptive region $Q_i$ for each task.
        \end{varwidth}
        \State\begin{varwidth}[t]{\linewidth}
        Select optimal preemption points using our \par
        \hskip\algorithmicindent dynamic programming algorithm.
        \end{varwidth}
        \State\begin{varwidth}[t]{\linewidth}
        Compute the CRPD and the preemptive WCET \par
        \hskip\algorithmicindent $C_i$ from the selected preemption points.
        \end{varwidth}
    \Until{the selected preemption points do not change.}
%
%    \If{the task system is feasible/schedulable}
%        \State\begin{varwidth}[t]{\linewidth}
%        Increase the system utilization by decreasing the \par
%        \hskip\algorithmicindent periods via a binary search.
%        \end{varwidth}
%    \Else
%        \State\begin{varwidth}[t]{\linewidth}
%        Decrease the system utilization by increasing the \par
%        \hskip\algorithmicindent periods via a binary search.
%        \end{varwidth}
%    \EndIf
%\Until\begin{varwidth}[t]{\linewidth}
%the system utilization change is less than some \par
%\hskip\algorithmicindent tolerance.
%\end{varwidth}
%\State{The breakdown utilization is given by U.}
\normalsize
\end{algorithmic}
\end{algorithm}
}
%\end{spacing}
\vspace{-10pt} 
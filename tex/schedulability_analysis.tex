\vspace{-5pt}
\section{Integrated WCET/CRPD Calculation}\label{sec:schedulability_analysis}

The modified preemption cost as a function of the current and next preemption points is given by:
\begin{equation}\label{eqn:prempt-cost}
\begin{split}
    \xi_{i}(\delta_{i}^{curr},\delta_{i}^{next})\ =\ &\gamma_{i}(\delta_{i}^{curr},\delta_{i}^{next}) + \pi + \sigma + \\ &\eta(\gamma_{i}(\delta_{i}^{curr},\delta_{i}^{next})).
\end{split}
\end{equation}
%
\noindent
where \begin{math}\pi\end{math} is the pipeline cost, \begin{math}\sigma\end{math} is the scheduler processing cost, and \begin{math}\eta()\end{math} is the front side bus contention resulting from the cache reload interference as described in~\cite{pellizzoni:07,pellizzoni:08,pellizzoni:11}.
%Commensurate with our worst case analysis, the preemption overhead cost as a function of the current and next %selected preemption points is:
%\begin{equation}\label{eqn:prempt-cost}
%    \xi_{i}(\delta_{i}^{curr},\delta_{i}^{next}) = \textit{max}_{k \in hp(i)} [ %\xi_{i}(\delta_{i}^{curr},\delta_{i}^{next},\tau_{k})].
%\end{equation}
%
The output of our algorithm is an optimal set of preemption points, with respect to our computed preemption cost $\xi_i()$, subject to the maximum allowable non-preemption region \begin{math}Q_{i}\end{math}.  The optimal set of preemption points obtained using the enhanced accuracy of our preemption cost computation is used to calculate each task's WCET with preemption overhead given by:
\vspace{-5pt}
\begin{equation}\label{eqn:wcet-cost}
   C_{i} = B_{i}(\rho_{i}) = C_{i}^{NP} + \sum_{m=1}^{|\rho_{i}|-1} [\xi_{i}(\rho_{i}^{m},\rho_{i}^{m+1})]
\end{equation}
\noindent
%where $\rho_i \subseteq \{\delta_i^0, \delta_i^1,  \ldots, \delta_i^{N_i}\}$ is an ordered set by ascending %index of selected preemption points for task \begin{math}\tau_{i}\end{math}:
%%\begin{math}\rho_{i}\end{math} is the set of selected preemption points for task
%\begin{equation*}\label{eqn:pp-set}
%\begin{split}
%   \rho_{i} \stackrel{\text{def}}{=} \{\delta_{i}^{m}|&\delta_{i}^{m} \text{ is a selected preemption point of %task } \tau_{i}\ \\ &\wedge m \in [0, %1, 2, \ldots, N_{i}]\}
%\end{split}
%\end{equation*}
%%
%\noindent
where \begin{math}\rho_{i}^{m}\end{math} is the \textrm{$m^{th}$} selected preemption point for task \begin{math}\tau_{i}\end{math}.  To further clarify what we mean by preemption, we say that $\delta_i^m$ is a ``preemption point'' if the preemption between basic blocks $\delta_i^m$ and $\delta_i^{m+1}$ is enabled, meaning the scheduler may preempt between these two basic blocks.  Commensurate with our preemption point placement algorithm discussed later, preemptions are always taken at basic blocks $\delta_i^0$ and $\delta_i^{N_i}$ hence $\delta_i^0,\ \delta_i^{N_i}\ \in\ \rho_{i}$ for any feasible set of preemption points $\rho_i$.  Supplemental clarification of this requirement is shown in the example of LCB interdependence in the appendix.
%\begin{equation*}\label{eqn:pp-mth}
%\begin{split}
%   \rho_{i}^{m} \stackrel{\text{def}}{=} \{\delta_{i}^{j}|&\delta_{i}^{j} \text{ is the $m^{th}$ selected %preemption point of task } \tau_{i}\ \\ &\wedge j \in [0, 1, 2, \ldots, N_{i}]\}
%\end{split}
%\end{equation*}
%
\noindent
The complete problem formulation with constraints for finding the minimum WCET with preemption overhead cost \begin{math}B_{i}(\rho_{i})\end{math} for task \begin{math}\tau_{i}\end{math} is given by:
\begin{equation}\label{eqn:global-bbkwcet-cost}
\begin{split}
   B_{i}(\rho_{i}) = \min_{\rho_{i} \in \tau_{i}} \Big\{&\Big[\sum_{m=1}^{|\rho_{i}|-1} \xi_{i}(\rho_{i}^{m},\rho_{i}^{m+1}) + \sum_{s=1}^{N_i}b_{i}^{s}\Big]\ |\\
   &\Psi_{i}(\rho_{i})\ =\ True\Big\}
\end{split}
\end{equation}
%
\noindent
The selection of optimal preemption points is subject to the constraint \begin{math}\Psi_{i}(\rho_{i})\end{math} that no non-preemptive region in task \begin{math}\tau_{i}\end{math} exceeds the maximum allowable non-preemption region parameter \begin{math}Q_{i}\end{math}:
\begin{equation}\label{eqn:global-pp-constraint}
   \Psi_{i}(\rho_{i}) =
\left\{
\begin{array}{lr}
    \textrm{True, }&\textrm{if } q_{i}^{m}(\rho_{i}) \leq Q_{i} \textrm{ for } m \in [1,|\rho_{i}|-1] \\
    \textrm{False, }&\textrm{otherwise}
\end{array}
\right\}~
\end{equation}
%
\noindent
where \begin{math}q_{i}^{m}(\rho_{i})\end{math} represents the \begin{math}m^{th}\end{math} non-preemptive-region (NPR) time for task \begin{math}\tau_{i}\end{math}, capturing the cost of the preemption $\rho_i^m$ given that $\rho_i^{m+1}$ is the next selected preemption point, plus the basic block cost of all blocks between $\rho_i^m$ and $\rho_i^{m+1}$:
\begin{equation}\label{eqn:global-mthnpr-time}
   q_{i}^{m}(\rho_{i}) = \Big[\xi_{i}(\rho_{i}^{m},\rho_{i}^{m+1}) +
   \sum_{s: \delta_i^s \in \{\rho_i^m,\ldots, \rho_i^{m+1}\}} b_{i}^{s}\Big]
\end{equation}
\noindent
%\newline
Our approach employs the results of schedulability analysis and the aforementioned WCET + CRPD calculation with the maximum allowable non-preemption region parameter \begin{math}Q_{i}\end{math} computed for each task \begin{math}\tau_{i}\end{math}.  The objective is to select a subset of preemption points that minimizes each tasks WCET + CRPD parameter \begin{math}C_{i}\end{math}. We introduce slight variations of terms \begin{math}\Psi_{i}\end{math}, \begin{math}q_{i}\end{math}, and \begin{math}B_{i}\end{math}, used in solving intermediate sub-problems of our proposed algorithm.  The selection of optimal preemption points is subject to the constraint that no non-preemptive region in task \begin{math}\tau_{i}\end{math} exceeds the maximum allowable non-preemption region parameter \begin{math}Q_{i}\end{math}:
\begin{equation}\label{eqn:pp-constraint}
   \Psi_{i}(\delta_{i}^{j},\delta_{i}^{k}) =
\left\{
\begin{array}{lr}
    \textrm{True, }&\textrm{if } q_{i}(\delta_{i}^{j},\delta_{i}^{k}) \leq Q_{i} \textrm{ for } \delta_{i}^{j},\delta_{i}^{k} \in \rho_{i} \\
    \textrm{False, }&\textrm{otherwise}
\end{array}
\right\}~
\end{equation}
\noindent
where \begin{math}q_{i}(\delta_{i}^{j},\delta_{i}^{k})\end{math} represents a possible candidate optimal non-preemptive-region (NPR) time with successive preemption points at basic block locations \begin{math}\delta_{i}^{j}\end{math} and \begin{math}\delta_{i}^{k}\end{math} for task \begin{math}\tau_{i}\end{math}:
\begin{equation}\label{eqn:mthnpr-time}
   q_{i}(\delta_{i}^{j},\delta_{i}^{k}) = \Big[\xi_{i}(\delta_{i}^{j},\delta_{i}^{k}) + \sum_{n=j+1}^{k}b_{i}^{n}\Big]
\end{equation}
\noindent
The next expression gives a recursive solution to the CRPD+WCET minimization problem for the subproblem of the first $i$ basic blocks.  We compute the WCET including the preemption point $\delta_i^k$ using the term \begin{math}B_{i}(\delta_{i}^{k})\end{math} for task \begin{math}\tau_{i}\end{math} as given by:
%We propose an O(N!) recursive relation shown in equation (\ref{eqn:bbkwcet-cost}) that computes the WCET %including preemption cost used in the dynamic programming algorithm for selecting the optimal preemption %points. While the recursive formulation is clearly inefficient, it is helpful in developing an understanding of %how the algorithm works. We compute the WCET including preemption cost for each potential preemption point %\begin{math}\delta_{i}^{k}\end{math} using the term \begin{math}B_{i}(\delta_{i}^{k})\end{math} for task %\begin{math}\tau_{i}\end{math} as given by:
%\begin{equation}\label{eqn:bbkwcet-cost-old}
%\begin{split}
%   B_{i}(\delta_{i}^{k}) = \min_{m \in \{0,1, \ldots, k-1\}}
%   \Big\{\Big[&B_{i}(\delta_{i}^{m}) + q_{i}(\delta_{i}^{m},\delta_{i}^{k})\Big]\ |\\
%   &\Psi_{i}(\delta_{i}^{m},\delta_{i}^{k})\ =\ True\Big\}
%\end{split}
%\end{equation}
%
\begin{equation}\label{eqn:bbkwcet-cost}
\begin{split}
   B_{i}(\delta_{i}^{k}) = \min_{m \in \{0,1, \ldots, k-1\}}
   \Big\{\Big[&B_{i}(\delta_{i}^{m}) + q_{i}(\delta_{i}^{m},\delta_{i}^{k})\Big]\ |\\
   &\Psi_{i}(\delta_{i}^{m},\delta_{i}^{k})\ =\ True\Big\}
%\wedge\\&k\ \in\ \{1, 2, \ldots, N_i\}\Big\}
\end{split}
\end{equation}
%\newline
\noindent
when $k\ \in\ \{1, 2, \ldots, N_i\}$.  For the base case where $\textrm{k = 0}$, we have $B_{i}(\delta_{i}^{0}) = 0$.  This recursive formulation is necessary for developing the dynamic programming solution presented in the next section.  The following theorem shows our problem has optimal substructure, and thus the formulation of Equation~\ref{eqn:bbkwcet-cost} represents an optimal solution.
%In accordance with the recursive nature of equation (\ref{eqn:bbkwcet-cost}) we propose an O(N!) recursive %algorithm shown in Algorithm~\ref{alg:recursive-optimal-ppp} for computing the optimal preemption points.  %While the recursive formulation is clearly inefficient, it is helpful in developing an understanding of how the %algorithm works.
%\noindent
%\begin{equation}\label{eqn:global-pp-constraint}
%   \Psi_{i}(\rho_{i}) =
%\left\{
%\begin{array}{lr}
%    \textrm{True, }&\textrm{if } q_{i}^{m}(\rho_{i}) \leq Q_{i} \textrm{ for } m \in [1,|\rho_{i}|-1] \\
%    \textrm{False, }&\textrm{otherwise}
%\end{array}
%\right\}~
%\end{equation}
%
%\noindent
%where \begin{math}q_{i}^{m}(\rho_{i})\end{math} represents the \begin{math}m^{th}\end{math} %non-preemptive-region (NPR) time for task \begin{math}\tau_{i}\end{math}:
%\begin{equation}\label{eqn:global-mthnpr-time}
%   q_{i}^{m}(\rho_{i}) = \Big[\xi_{i}(\rho_{i}^{m},\rho_{i}^{m+1}) + %\sum_{n=\rho_{i}^{m}}^{\rho_{i}^{m+1}}b_{i}^{n}\Big]
%\end{equation}
%
%\noindent
%In accordance with the recursive nature of equation (\ref{eqn:bbkwcet-cost}) we propose an O(N!) recursive %algorithm shown in Algorithm~\ref{alg:recursive-optimal-ppp} for computing the optimal preemption points.  %While the recursive formulation is clearly inefficient, it is helpful in developing an understanding of how the %algorithm works.  Starting with the first basic block and for each successive basic block %\begin{math}\delta_{i}^{m}, m \in [1,N_{i}]\end{math}, the overall WCET cost is computed for two cases: 1) %\begin{math}\delta_{i}^{m} \in \rho_{i}\end{math}, and 2) \begin{math}\delta_{i}^{m} \not\in %\rho_{i}\end{math}.  At each step of the algorithm the set \begin{math}\rho_{i}\end{math} must conform to the %constraint of equation (\ref{eqn:pp-constraint}).  Once basic block \begin{math}\delta_{i}^{N_{i}}\end{math} %has been examined, the set of selected preemption points is given by \begin{math}\rho_{i} = %\rho_{i}^{(N_{i})}\end{math}.  The WCET cost with basic block \begin{math}\delta_{i}^{m}\end{math} included in %the set of potential preemption points is given by:
%\begin{equation}\label{eqn:pcost-bb}
%p_{COST}(\delta_{i}^{m})\ =\ B_{i}^{0,m}(\rho_{i}) + B_{i}^{m,N_{i}}(\rho_{i})
%\end{equation}
%
%The WCET cost with basic block \begin{math}\delta_{i}^{m}\end{math} excluded from the set of potential %preemption points is given by:
%\begin{equation}\label{eqn:npcost-bb}
%n_{COST}(\delta_{i}^{m})\ =\ B_{i}^{0,m-1}(\rho_{i}) + b_{i}^{m} + B_{i}^{m+1,N_{i}}(\rho_{i})
%\end{equation}
%%
%\noindent
%where \begin{math}B_{i}^{k,m}\end{math} is the WCET with preemption overhead of basic blocks $k$ to $m$ of task %\begin{math}\tau_{i}\end{math} as given by:
%\begin{equation}\label{eqn:global-bbkwcet-cost}
%\begin{split}
%   B_{i}^{k,m}(\rho_{i}) = \min_{\delta_{i}^{j} \in \tau_{i}} \Big[&B_{i}^{j-1}(\rho_{i}(\delta_{i}^{j-1})) + %\\ &\xi_{i}(\delta_{i}^{j},\rho_{i}^{next}(\delta_{i}^{j})) + \sum_{n=k}^{m}b_{i}^{n}\Big]
%\end{split}
%\end{equation}
%%
%\noindent
%where the term \begin{math}\rho_{i}^{next}(\delta_{i}^{j})\end{math} is the next potential/selected preemption %point from basic block \begin{math}\delta_{i}^{j}\end{math}:
%\begin{equation}\label{eqn:ppnext-set}
%\begin{split}
%   \rho_{i}^{next}(\delta_{i}^{k}) \stackrel{\text{def}}{=} \{\delta_{i}^{j}|&\delta_{i}^{k} = \rho_{i}^{m} %\wedge \delta_{i}^{j} = \rho_{i}^{m+1} \\ &\textrm{ for some } m \in [1,N_{i}-1]\}
%\end{split}
%\end{equation}
%{\fontsize{10}{10}\selectfont
%\begin{algorithm}
%\caption{Recursive Optimal Preemption Point Placement}
%\label{alg:recursive-optimal-ppp}
%\begin{algorithmic}[0]
%\small
%\State \textbf{\underline{Step 0:}}
%%\\
%\State {$\ \ \rho_{i}^{(0)}\ \gets\ \{\delta_{i}^{0},\delta_{i}^{N_{i}}\}$};
%\State {$\ \ \textbf{if}\ \Psi_{i}^{N_{i}}(\{\delta_{i}^{0},\delta_{i}^{1},\delta_{i}^{2}, ..., %\delta_{i}^{N_{i}}\}) = False$\ \textbf{then}}
%\State {$\ \ \ \ \ \ \textbf{return}\ \textbf{INFEASIBLE;}$}
%\State {$\ \ \textbf{end if}$}
%\\
%\State \textbf{\underline{Step 1:}}
%\begin{equation*}
%\rho^{(1)}\ \gets\
%\left\{
%\begin{array}{l}
%    \rho^{(0)}, if\ n_{COST}(\delta_{i}^{1}) < p_{COST}(\delta_{i}^{1})\ \&\ \ \ \ \ \ \ \ \ \ \ \\
%    \ \ \ \ \ \ \ \ \ \ \Psi_{i}^{1}(\rho^{(0)}) = True\\
%    \rho^{(0)}\ \cup\ \delta_{i}^{1},\ otherwise
%\end{array}
%\right\}~
%\end{equation*}
%\State \textbf{\underline{Step 2:}}
%\begin{equation*}
%\rho^{(2)}\ \gets\
%\left\{
%\begin{array}{l}
%    \rho^{(1)}, if\ n_{COST}(\delta_{i}^{2}) < p_{COST}(\delta_{i}^{2})\ \&\ \ \ \ \ \ \ \ \ \ \ \\
%    \ \ \ \ \ \ \ \ \ \ \Psi_{i}^{2}(\rho^{(1)}) = True\\
%    \rho^{(1)}\ \cup\ \delta_{i}^{2},\ otherwise
%\end{array}
%\right\}~
%\end{equation*}
%\ldots
%\State \textbf{\underline{Steps m = 1,\ \ldots,\ $N_{i}$:}}
%\begin{equation*}
%\rho_{i}^{(m)}\ \gets\
%\left\{
%\begin{array}{l}
%    \rho_{i}^{(m-1)}, if\ n_{COST}(\delta_{i}^{m}) < p_{COST}(\delta_{i}^{m})\ \&\ \ \ \ \\
%    \ \ \ \ \ \ \ \ \ \ \ \ \ \Psi_{i}^{m}(\rho_{i}^{(m-1)}) = True\\
%    \rho_{i}^{(m-1)}\ \cup\ \delta_{i}^{m},\ otherwise
%\end{array}
%\right\}~
%\end{equation*}
%\ldots
%\State \textbf{\underline{Step $N_{i}$:}}
%\begin{equation*}
%\rho^{(N_{i})}\ \gets\
%\left\{
%\begin{array}{l}
%    \rho^{(N_{i}-1)}, if\ n_{COST}(\delta_{i}^{N_{i}}) < p_{COST}(\delta_{i}^{N_{i}})\ \&\\
%    \ \ \ \ \ \ \ \ \ \ \ \ \ \ \Psi_{i}^{N_{i}}(\rho^{(N_{i}-1)}) = True\\
%    \rho^{(N_{i}-1)}\ \cup\ \delta_{i}^{N_{i}},\ otherwise
%\end{array}
%\right\}~
%\end{equation*}
%\normalsize
%\end{algorithmic}
%\end{algorithm}
%}
%
\begin{theorem}
\label{thm:optimal-substructure-cost}
The WCET + CRPD cost variable $B_{i}(\delta_{i}^{k})$ utilized in Equation~\ref{eqn:bbkwcet-cost} exhibits optimal substructure.
\end{theorem}
%
%\noindent
%We now prove the optimal substructure property of the WCET + CRPD cost $B_{i}(\delta_{i}^{j},\delta_{i}^{k})$.
%\newline
\noindent
\begin{proof}
%Let $B_{i}(\delta_{i}^{k})$ with its corresponding selected preemption points denoted by $\rho_{i}^{k}$ be the optimal limited preemption execution cost solution from basic block $\delta_{i}^{0}$ to basic block $\delta_{i}^{k}$, and assume the optimal cost solution contains the basic block identified by $\delta_{i}^{m}$. Furthermore, let $B_{i}(\delta_{i}^{m})$ with selected preemption points denoted by $\rho_{i}^{m}$ and $q_{i}(\delta_{i}^{m},\delta_{i}^{k})$ be the optimal limited preemption execution cost from $\delta_{i}^{0}$ to $\delta_{i}^{m}$, respectively, contained in the optimal solution to the original problem. Three additional constraints are imposed where $\rho_{i}^{m} \subseteq \rho_{i}^{k}$ with $B_{i}(\delta_{i}^{m}) + q_{i}(\delta_{i}^{m},\delta_{i}^{k}) = B_{i}(\delta_{i}^{k})$.
Let $\Delta_i^k$ be the subproblem of the sequential control flowgraph containing basic blocks $\{\delta_i^0, \delta_i^1, \ldots, \delta_i^{k}\}$.
To prove optimal substructure, we show that we can obtain the optimal set of preemption points for minimizing the WCET+CRPD for any $\Delta_i^k$ by using the optimal solutions to subproblems $\Delta_i^0, \Delta_i^1, \ldots, \Delta_i^{k-1}$.  Let $B_i(\delta_i^0), B_i(\delta_i^1), \ldots, B_i(\delta_i^{k-1})$ represent the cost of the optimal solution to these subproblems.  We need to show that Equation~\ref{eqn:bbkwcet-cost} represents the optimal cost to $\Delta_i^k$.

By way of contradiction, assume there is a better feasible solution $\rho_{i}'$ for the sub-problem of determining the optimal limited preemption execution costs from basic block $\delta_{i}^{0}$ to basic block $\delta_{i}^{k}$; that is, $B_{i}(\rho_{i}')$ is strictly smaller than the solution to $\Delta_i^k$ obtained in Equation~\ref{eqn:bbkwcet-cost} (i.e.,  $B_{i}(\delta_{i}^{k})$).  Let $\delta_i^{\ell}$ (where $\ell \in \{0, 1, \ldots, k-1\}$) be the last preemption point prior to $\delta_i^k$ in the set $\rho_{i}'$, and let $\rho_{i}''$ be the set of preemption points obtained from $\rho_{i}'$ by removing the preemption point after $\delta_i^k$ (i.e., $\rho_{i}''$ is a solution to $\Delta_i^{\ell}$).  Thus, we can represent the cost of the solution $\rho_{i}'$ (i.e., $B_i(\rho_{i}')$ by the left-hand-side of the following inequality:
\begin{equation}\label{eqn:bi-deltai-lb}
    B_i(\rho_{i}'') + q_i(\delta_i^{\ell}, \delta_i^k) < B_i(\delta_i^k).
\end{equation}

Since $\rho_{i}'$ is a feasible solution to $\Delta_i^k$, it must be that $\Psi_i(\delta_i^{\ell}, \delta_i^k)$ is true.  Hence, from Equation~\ref{eqn:bbkwcet-cost} and the $\min$ operation, we can obtain an upper bound on $B_i(\delta_i^k)$ by considering the solution to subproblem $\Delta_i^{\ell}$:
\begin{equation}\label{eqn:bi-deltai-ub}
    B_i(\delta_i^k) \leq B_i(\delta_i^{\ell}) + q_i(\delta_i^{\ell}, \delta_i^k).
\end{equation}

Combining the inequalities of Equations~\ref{eqn:bi-deltai-lb} and~\ref{eqn:bi-deltai-ub}, we finally obtain $B_i(\rho_{i}'') < B_i(\delta_i^{\ell})$.  However, this contradicts our assumption at the beginning of the proof that $B_i(\delta_i^{\ell})$ represented an optimal solution to subproblem $\Delta_i^{\ell}$.  Thus, a solution $\rho_{i}'$ with smaller cost than $B_i(\delta_i^k)$ cannot exist, and Equation~\ref{eqn:bbkwcet-cost} computes the minimum obtainable cost for the problem $\Delta_i^k$.
%Since the solution for selecting the optimal preemption points from basic block $\delta_{i}^{0}$ to basic block $\delta_{i}^{k}$ will work regardless of which set of preemption points are used, we could then use the better solution  $B_{i}^{'}(\delta_{i}^{m})$ and $\rho_{i}^{'m}$ to arrive at a lower cost solution to the original problem. Thus we have the following:
%\begin{equation}\label{eqn:pcost-bb-1}
%B_{i}(\delta_{i}^{k})\ =\ B_{i}(\delta_{i}^{m}) + q_{i}(\delta_{i}^{m},\delta_{i}^{k})
%\end{equation}
%\begin{equation}\label{eqn:pcost-bb-2}
%B_{i}(\delta_{i}^{k})\ >\ B_{i}^{'}(\delta_{i}^{m}) + q_{i}^{'}(\delta_{i}^{m},\delta_{i}^{k})
%\end{equation}
%\noindent
%However this contradicts the original definition that $B_{i}(\delta_{i}^{k})$ and $\rho_{i}^{k}$ form an optimal solution to the problem. Therefore, the optimal limited preemption execution cost from basic block $\delta_{i}^{0}$ to basic block $\delta_{i}^{k}$ contained in the original solution $B_{i}(\delta_{i}^{k})$ and $\rho_{i}^{k}$ must be an optimal solution to the sub-problem of determining the optimal limited preemption execution cost solution from basic block $\delta_{i}^{0}$ to basic block $\delta_{i}^{k}$. Thus, the problem exhibits optimal substructure.
\end{proof}

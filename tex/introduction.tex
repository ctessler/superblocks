\section{Introduction}\label{sec:introduction}

%introduction
Hard-real-time systems differ from traditional computing systems in that one or more tasks must complete execution within a specified deadline, otherwise the utility of the result is severely diminished or utterly useless.  Ensuring hard-real-time task systems meet their corresponding deadlines is the subject of prominent research on schedulability analysis.  Paramount to schedulability analysis is accurate characterization of worst case execution time (WCET).  While the debate continues on whether non-preemptive versus fully preemptive execution is more effective, recent research on limited preemption models has shown to outperform non-preemptive and fully preemptive scheduling approaches in terms of the number of preemptions and schedulability when preemption cost is considered.  Non-preemptive execution has the disadvantage of introducing blocking of high priority tasks by lower priority tasks whereas fully preemptive execution has the disadvantage of subjecting preemption overhead which has shown to be up to 44\%~\cite{pellizzoni:07}~\cite{pellizzoni:08}~\cite{pellizzoni:11} of a tasks worst case execution time (WCET).  One of the prominent contributors to preemption overhead is due to cache related preemption delay (CRPD).  CRPD occurs when a task denoted \begin{math}\tau_{i}\end{math} is preempted by one or more higher priority tasks denoted \begin{math}\tau_{k}\end{math} whose execution results in the eviction of cache memory blocks that must be subsequently reloaded when task \begin{math}\tau_{i}\end{math} resumes execution.  Limited preemption approaches have the advantage of reduced blocking with a limited number of allowed preemptions while having the advantage of sections of non-preemptive regions (NPRs) reducing the preemption overhead.  One promising approach to implementing a limited preemption approach is selecting preemption points for each task subject to the constraint on maximum non-preemptive region execution time \begin{math}Q_{i}\end{math}.  A recent paper by Bertogna et. al.~\cite{bertogna:11} proposed and realized a polynomial time algorithm for selecting optimal preemption points for a linear basic block structure.  A linear basic block structure implies that conditional logic and branches are fully contained within basic block boundaries. CRPD literature permits basic blocks connected by arbitrary control flow graph structures.  The figure shown below illustrates the problem scope for selecting optimal preemption points for a linear basic block structure.

%outline of paper
The rest of this paper is organized as follows. First, current research efforts and related work in the areas of cache related preemption delay (CRPD) and limited preemption scheduling is discussed in Section~\ref{sec:related}.  Section~\ref{sec:system_model} describes the real-time task model terminology used in this paper.  The enhanced CPRD computation approach is detailed in Section~\ref{sec:crpd_computation}. Section~\ref{sec:schedulability_analysis} briefly outlines the impact of the revised CRPD computation approach incorporated into fixed priority (FP) and earliest deadline first (EDF) schedulability analysis.  The enhanced limited preemption point placement algorithm leverages the improved CPRD computation in a polynomial time algorithm as discussed in Section~\ref{sec:implementation}.  Synthetic task schedulability comparison of various preemption models in addition to a case study using well known bench mark tasks are presented and summarized in Section~\ref{sec:evaluation}.  Finally we will offer relevant conclusions along with proposed future work in Section~\ref{sec:conclusion}. 
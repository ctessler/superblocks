\section{Conclusion}\label{sec:conclusion}

In this paper we presented an enhanced approach for calculating the 
CRPD taking into account the selected preemption points resulting 
in greater accuracy.  Using the enhanced CRPD calculation, we also
presented an improved algorithm for selecting a limited number of 
preemption points for each task subject to schedulability constraints.
Our improved preemption placement algorithm was demonstrated to
minimize the overall preemption cost, an important result in achieving
schedulability in real-time system.  We highlighted the iterative nature
of considering schedulability constraints in our preemption point 
placement algorithm and proposed an algorithm combining schedulability
analysis with limited preemption point placement.  This approach
effectively illustrates how the individual tasks non-preemption
region parameters $Q_i$ and the optimal selected preemption points will  
eventually converge.  Furthermore, our enhanced algorithm was demonstrated to be 
optimal in that if a feasible schedule is not found, then no feasible 
schedule exists by any method.  Our algorithm was shown to run in quadratic time
complexity.  Potential preemption points can be defined automatically
using Gaisler's compiler and simulator along with AbsInt's
a\textsuperscript{3} tool or defined manually by the programmer during design
and implementation. Our experiments demonstrated the effectiveness of 
the enhanced CRPD calculation by illustrating the benefits using the 
task set from Malardalen University of Sweden's WCET benchmark suite~\cite{mrtc:01}.
We also demonstrated the benefits of our enhanced limited optimal preemption
point placement algorithm and its increased system schedulability as 
compared to other algorithms.  While our task model is defined using a linear 
sequence of basic blocks, it was deemed a highly suitable model to introduce
our revised methods for enhanced CRPD calculation and optimal limited
preemption point placement.  In future work, we plan to 1) a schedulability comparison
of synthetic task set for various preemption models, and 2) remove the linear
basic block restriction thereby permitting arbitrarily connected basic
block structures.
